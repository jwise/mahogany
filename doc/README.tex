%% This LaTeX-file was created by <karsten> Sun Nov 22 20:19:05 1998
%% LyX 1.0 (C) 1995-1998 by Matthias Ettrich and the LyX Team

%% Do not edit this file unless you know what you are doing.
\documentclass[12pt,english]{article}
\usepackage[T1]{fontenc}
\usepackage{palatino}
\usepackage{babel}

\makeatletter


%%%%%%%%%%%%%%%%%%%%%%%%%%%%%% LyX specific LaTeX commands.
\newcommand{\LyX}{L\kern-.1667em\lower.25em\hbox{Y}\kern-.125emX\spacefactor1000}

\makeatother

\begin{document}


\title{\textsl{M} - README}


\author{Karsten Ball�der \char`\\{}\char`\\{}(\texttt{Ballueder@usa.net})\char`\\{}\char`\\{}Contact
us: \texttt{mahogany-developers@lists.sourceforge.net}}

\maketitle
\begin{abstract}
These are the release notes for M. Some more detailed information on how to
compile, install und use it are in the Information files.

This information relates to the first public alpha release, 17. August 1998.
\end{abstract}

\section{WARNING - this is (pre-?)alpha}

When you do a large free software project, you have two choices: either continue
hacking without releases and risk to never get it finished, or set yourself
a release date, no matter what the situation is. As we got several requests
from people who want to play with it, we decided on a release date and did our
best to get it into a working shape. BUT THIS SOFTWARE IS STILL UNDER DEVELOPMENT!
This means

\begin{itemize}
\item it is incomplete and awkward to use
\item it may crash occasionally or often or be completely unusable - \textit{Use it
at your own risk!}
\end{itemize}
M is not ready for the end user yet, but we present it in its current state
to give you an impression of what it is going to be. We also hope to attract
a bit of attention and maybe even some outside help for it.


\section{Which features are implemented?}

Quite some already, but more are still missing. What we have so far:

\begin{itemize}
\item Cross-platform. M compiles on a variety of Unix systems and on Microsoft Windows.
Use one mail client, no matter what system you use. The source and binary for
Windows 95/98/NT are available on request (\texttt{mahogany-developers@lists.sourceforge.net}).
Mailbox file formats are the same on both platforms.
\item Based on the c-client library from the University of Washington, therefore full
access to a wide range of protocols and file formats, including SMTP, MAP, POP3,
NNTP and several mailbox formats.
\item Wide (extreme?) user configurability. Whatever makes sense to override or change,
can be changed by the user. Configuration supports several configuration files
on Unix, with special administrator support for making entries immutable, and
the registry on Windows.
\item Scriptable and extendable. M includes an embedded Python interpreter with full
access to its object hierarchy. Write object-oriented scripts to extend and
control M.
\item Easy MIME support. Text and other content can be freely mixed and different
filetypes are represented by icons.
\item Inline displaying of images, clickable URLs, XFace support.
\item Multiple mail folders.
\item Powerful address database and contact manager 
\item Printing of nicely formatted messages.
\item Full internationalisation support, M speaks multiple languages, but no translations
yet.
\end{itemize}

\section{Known bugs}

\begin{itemize}
\item Folder creation dialog doesn't work properly. See Information file for how to
setup new mail folders.
\item Selection in the folder view sometimes behaves strange, selecting all messages
doesn't work.
\item Message and composition view don't automatically scroll to the cursor.
\item Tab traversal in dialogs doesn't work (wxGTK problem).
\end{itemize}

\section{TODO, features to implement}

This is a list of features on our TODO list that we are currently working on.
Before adding new features, we'll clean up a few things:

\begin{itemize}
\item First comes a rewrite of the class hierarchy. For better modularisation and
CORBA support (Python will profit from this, too.), we will clean up header
files and remove some interdependencies. GUI and non-GUI code will be better
separated, class implementations and interface definitions will be sorted out
more clearly. This includes a common base object with reference couting.
\item Plug the (apparently very few remaining) memory holes.
\end{itemize}
Then we fix some GUI issues. Many of these depend on wxGTK which is still evolving
speedily. 

\begin{itemize}
\item add keyboard accelerators and proper tab traversal
\item add a context sensitive help system
\item add more dialogs and a tree control for folder selection
\end{itemize}
After that we reach the list of serious improvements:

\begin{itemize}
\item Better Python support. We have some callbacks in place, but after the class
hierarchy rewrite we have to generate new interface files for the complete class
hierarchy. Also by this time wxPython might be integrated, so we can actually
write some of the configuration dialogs in python which should speed things
up. \textit{Help welcome.}
\item Full Drag and Drop interaction with filemanagers of Windows and Gnome (will
be added real soon, easy).
\item Easy to use filtering system for mails.
\item Support for V-cards.
\item Nested mail folder hierarchy.
\item Spam-Ex spam fighting/auto-complaint function.
\item Richt-text editing and HTML mail support
\item Support for PGP and GNU Privacy Guard to encrypt mails.
\item Threading of messages and proper usenet news support.
\item Compression of mail folders.
\item Delay-Folder to keep mails and re-present them at a later date.
\item Context sensitive help system (HTML based).
\item Translations to German, French and Italian.
\item Wide character (Unicode) support and other character sets.
\item Import, export and synchronisation with other programs' address databases.
\item Voice mail.
\item More Python support through wxPython.
\item Support for Drag and Drop interaction with KDE, once that wxQt is available.
\item CORBA support, possible cooperation with PINN project.
\item Address datbase synchronisation with PDA's (Just got one...)
\item ANY OTHER SUGGESTION
\end{itemize}

\subsection{Help Needed}

As you can see, we have big plans for M. To achieve all this, we need some help.
Areas where we would use some help are

\begin{itemize}
\item Python 
\item support for further mail protocols, LDAP
\item The wxQt project, a port of wxWindows to the Qt toolkit, will also be happy
for any help. We are not directly involved in this, but being involved with
wxWindows, we are happy to support that port.
\item If you have access to other systems apart from Linux/Solaris/Windows, you are
very welcome to help us port M to those platforms, or to other hardware than
Intel.
\end{itemize}

\section{Online resources}

\begin{itemize}
\item M has a homepage at \texttt{http://www.wxwindows.org/Mahogany/} 
\item The wxWindows homepage is \texttt{http://www.wxwindows.org/}
\item wxGTK, the GTK port of wxWindows, is available from \texttt{http://www.wxwindows.org/dl_gtk.htm}
\end{itemize}

\section{FAQ}

There will be some after this release - surely.

\end{document}
